\documentclass[10pt,a4paper]{ctexart}
\usepackage{ctex}
\usepackage{titlesec}%调节标题格式
\titleformat*{\section}{\zihao{3}\bfseries\centering}
\titleformat*{\subsection}{\zihao{4}\bfseries}
\titleformat*{\subsubsection}{\zihao{4}\bfseries}
\usepackage[utf8]{inputenc}
\usepackage[T1]{fontenc}
\usepackage{amsmath}
\usepackage{pifont}
\renewcommand\thefootnote{\ding{\numexpr171+\value{footnote}}}
\newcommand\Emph{\textbf}
\usepackage{amssymb}
\usepackage{makeidx}
\usepackage{graphicx}
\graphicspath{{figures/}}%figures环境
\usepackage{float}
\usepackage{booktabs}
\usepackage{multirow}
\usepackage[section]{placeins}
\usepackage{geometry}
\geometry{hcentering}
\geometry{left=2cm,right=2cm,top=3cm,bottom=3cm}%设置左右页边距
\setcounter{secnumdepth}{4}%设置目录标题深度为4
\setcounter{tocdepth}{4}
\usepackage{color}
\usepackage{tabularray}
\usepackage{hyperref}
\begin{document}
	\begin{titlepage}
\centering
\zihao{1}鼠鼠的材力期末\protect\footnotemark%脚注符号与内容分开,这是脚注符号
\\
\vspace*{\baselineskip}
\zihao{-4}by:一只学土木的鼠鼠
	\footnotetext{edited by \LaTeX}%脚注符号与内容分开,这是脚注内容
	\newpage
	\end{titlepage}
 %\setlength{\parindent}{0pt}%取消所有段落首行缩进
 \noindent%取消该段首行缩进
 \zihao{1}前面的话

\vspace*{1ex}%设置段落间的垂直间距
\zihao{-4}这个材料力学的期末小册子是基于孙训方的《材料力学(\uppercase\expandafter{\romannumeral1}%这罗马数字也太tm难打了吧
 )》(第五版)\cite{孙训方2009}
 整理出来的,纯粹是鼠鼠为了期末复习和顺便重拾\LaTeX 而整理,如有真正想学好材料力学的朋友建议啃书本,鼠鼠只是想期末及格。
 
 而且这玩意儿我预感会通篇公式,几乎没有图片。本来鼠鼠是打算给必要的地方放上Axglyph、PowerPoint等软件画的矢量图,但实际操作还是不熟悉。等后面鼠鼠有时间了会捣鼓捣鼓\LaTeX 写论文的工作流,会给这篇加上矢量图的。
\thispagestyle{empty}%不标出页码
 \newpage
\tableofcontents
\setcounter{page}{0}%页码标记为0
\thispagestyle{empty}
\newpage
\setcounter{secnumdepth}{-1}%设置section章节不标号
\section{第一章:绪论及基本概念}
PS:经典大学课本第一章绪论。鼠鼠暂时没看到有什么重点,看到了鼠鼠会补充
\section{第二章:轴向拉伸和压缩}
这章可以说是材料力学中基础中的基础,第二节的截面法求轴力和画截面图更是后面所有内容的根基。
\subsection{轴力及轴力图}
轴力即作用线与杆轴线重合的力。在我们所学的范围内,杆件所受的拉力和压力绝大多数为轴力。\textcolor{red}{符号规定:}拉力为正,压力为负,一般计算先设为正,如果计算为负即为压力。

\textcolor{cyan}{截面法画轴力图:}三步走:截开、代替(用内力代替弃去部分)、平衡
可能会出大题,比较简单:先整体受力分析求支反力(求外力),再用截面法求内力(有时候右边比左边好算,具体情况具体分析),作图即可。
\subsection{应力}
应力分两种:正应力和切应力,正应力引发长度的改变,切应力引发角度的改变。本章主要是正应力的计算,其实可以完全把正应力当作压强看待,其计算公式为
\begin{equation}
	\sigma=\frac{F_N}{A}
	\label{zyl}
\end{equation}
单位为Pa,其中A为横截面面积。符号与轴力保持一致(拉正压负)。

切引力的\textcolor{red}{符号规定}要注意:对截面一点产生顺时针力矩的切应力为正,反之为负。

当发生\textcolor{red}{轴向拉压}时,斜截面上的正应力和切应力有公式解决,先如图示(不得不画图了)
\begin{figure}[htp]%设置figure环境
	\centering
	\includegraphics[width = 8cm ]{xjm}
\end{figure}

公式如下

\begin{equation}
	\begin{cases}
		\sigma_\alpha=p_\alpha cos\alpha =\sigma_0 cos^2\alpha\\
		\tau_\alpha=p_\alpha sin\alpha  = \frac{\sigma_0}{2}sin2\alpha
	\end{cases}
\end{equation}

$\sigma_0=\frac{F}{A}$是拉杆在横截面($\alpha=0$)上的正应力,由公式知切应力的最大值在与轴线成$45^{\circ}$角的斜截面上,大小为$\tau_{max}=\frac{\sigma}{2}$
\subsection{许用应力}
简单说就是杆件所受最大应力不应超过许用应力,公式为
\begin{equation}
	\sigma_{max}\leqslant\left[\sigma\right]%数学环境打出括号
\end{equation}
\par 若为轴向拉压,加上正应力公式,即公式\ref{zyl}%交叉引用的命令是ref,别忘了
\begin{equation}
	\sigma_{max}=\frac{F_{N,max}}{A}\leqslant\left[\sigma\right]
\end{equation}
\par
由它可以得到强度计算的三种类型:

\begin{table}[h]
	\centering
	\begin{tblr}{|c|c|}%好好看看tabularray是怎么用的
		\hline
		强度校核 & $\sigma_{max}=\frac{F_{N,max}}{A}\leqslant\left[\sigma\right]$ \\
		\hline
		截面选择 & $A\geqslant\frac{F_{N,max}}{\left[\sigma\right]}$\\
		\hline
		计算许可负载 & $F_{N,max}=A\left[\sigma\right]$\\
		\hline
	\end{tblr}
\end{table}


\subsection{横向线应变、纵向线应变、泊松比及胡克定律}
别管别的,把横纵向的区别搞清楚先:
\begin{figure}[htp]%设置figure环境
	\centering
	\includegraphics[width = 6cm ]{hzx}
\end{figure}

\textcolor{cyan}{线应变}为单位长度上的伸长和缩短(压缩和膨胀)。由定义得纵向线应变$\varepsilon$和横向线应变$\varepsilon'$为
\begin{equation}
\varepsilon=\frac{\Delta l}{l} \qquad%两公式间间隔
  \varepsilon'=\frac{\Delta l}{l}
\end{equation}

线应变的\textcolor{red}{符号规定}要注意:纵向伸长为正反之为负;横向压缩为正反之为负。

\textcolor{cyan}{泊松比}(横向变形因数)为\textcolor{red}{横向线应变$\varepsilon'$}与\textcolor{red}{纵向线应变之$\varepsilon$}之比(别比反了!),数值随材料而异。
\begin{equation}
 \nu=\frac{\varepsilon'}{\varepsilon} \Rightarrow \varepsilon'=-\nu \varepsilon
 \label{psb}
\end{equation}

\textcolor{cyan}{胡克定律}的原始表达式为:
\begin{equation}
\Delta l=\frac{F_N l}{EA}
\label{hkdl}
\end{equation}

其中E被称作材料的弹性模量,数值随材料而定。EA被称作杆件的拉伸(压缩)刚度,是衡量杆件抗拉(压)变形能力的,越大越强。

由公式\ref{psb}、\ref{hkdl},得胡克定律的应力-应变表达式:
\begin{equation}
\sigma=E\varepsilon
\end{equation}
\subsection{弹性应变能及弹性应变能密度}
\textcolor{cyan}{弹性应变能}为弹性体所受力而变形时所积蓄的能量。表达式为:
\begin{equation}
	V_{\varepsilon}=\frac{F^2_N l}{2EA}
\end{equation}

联立公式\ref{hkdl},得:
\begin{equation}
	V_{\varepsilon}=\frac{EA}{2l}\Delta l^2
\end{equation}

\textcolor{cyan}{弹性应变能密度}为单位体积的弹性应变能,即:
\begin{equation}
	v_{\varepsilon}=\frac{V_{\varepsilon}}{V}=\frac{\frac{1}{2}F\Delta l}{Al}=\frac{1}{2}\sigma\varepsilon
\end{equation}

联立公式\ref{hkdl},得:
\begin{equation}
	v_{\varepsilon}=\frac{E\varepsilon^2}{2}=\frac{\sigma^2}{2E}
\end{equation}
	 
\subsection{材料在拉伸和压缩时的力学性能}
这节就两理解性的公式,后面全是过程叙述,鼠鼠捡几个重点稍微说说。

\textcolor{cyan}{伸长率}$\delta$和\textcolor{cyan}{断面收缩率}$\psi$的公式为
\begin{equation}
	\delta=\frac{l_1-l}{l}\times 100\% \qquad \psi=\frac{A-A_1}{A}\times 100\%
\end{equation}

通常$\delta>5\%$的材料被称为\textcolor{cyan}{塑性材料},$\delta<5\%$的材料被称为\textcolor{cyan}{脆性材料}。

力学性能为材料受力时在强度和变形方面所表现出来的性能。

在\textcolor{cyan}{低碳钢的拉伸实验}中,由应力-伸长量图像可将拉伸过程分为四个阶段,如图所示:
\begin{figure}[htp]%设置figure环境
	\centering
	\includegraphics[width = 8cm ]{dtgls}
\end{figure}

\uppercase\expandafter{\romannumeral1}为\textcolor{red}{弹性阶段},满足胡克定律,比例极限$\sigma_p$对应点A(请自行放大,确实不太清楚),弹性极限$\sigma_e$对应点B。两者区别在于弹性极限是在弹性范围内的,即形变可以完全恢复;但比例极限是有可能超过了弹性形变的范围,仅仅是数值上符合胡克定律\cite{匿名用户2017}。

\uppercase\expandafter{\romannumeral2}为\textcolor{red}{屈服阶段},此阶段应变显著增加,但应力基本不变(屈服现象)。屈服极限$\sigma_s$对应D点。

\uppercase\expandafter{\romannumeral3}为\textcolor{red}{强化阶段},此阶段材料抵抗变形的能力有所增强。强度极限$\sigma_b$对应点G,为最大名义应力。

\uppercase\expandafter{\romannumeral4}为\textcolor{red}{局部变形阶段},试件上出现急剧的局部横截面收缩现象(颈缩现象),直至试件断裂。

\textcolor{cyan}{冷作硬化现象}指在强化阶段卸载后,材料的比例极限提高,塑性降低的现象。过程不再详细叙述,只需记住这样后比例极限$\sigma_p$提高,塑性变形降低,强度极限$\sigma_b$不变。

后面(课本P33及以后)就是一堆没什么规律的文字叙述了。自己看看书,鼠鼠在这里不做赘述。

\subsection{应力集中(概念)}
\textcolor{cyan}{应力集中}指杆件突然变化而引起的应力局部骤然增大的现象。用\textcolor{cyan}{理论应力集中系数}$K_{t\sigma}$描述应力集中程度:
\begin{equation}
	K_{t\sigma}=\frac{\sigma_{max}}{\sigma_{nom}}
\end{equation}

其中$\sigma_{max}$为杆件外形局部不规则处的最大局部应力,$\sigma_{nom}$为横截面的平均应力。

考虑应力集中的情况如下表所示:
\begin{table}[h]
	\centering
	\begin{tblr}{|c|c|}%好好看看tabularray是怎么用的
		\hline
		不考虑应力集中 & 塑性材料、静荷载;非均匀的脆性材料,如铸铁\\
		\hline
		考虑应力集中 & 均匀的脆性材料或塑性差的材料;动荷载\\
		\hline	
	\end{tblr}
\end{table}
\newpage

\section{附录\uppercase\expandafter{\romannumeral1}:截面的几何性质}
本章涉及大量的公式推导及数学计算,鼠鼠为追求短平快在此只列出必要的定义、公式及结论。对公式推导感兴趣的话可以翻翻书本。

\subsection{静矩和形心位置}
先上图:
\begin{figure}[htp]%设置figure环境
	\centering
	\includegraphics[width = 6cm ]{jjxx}
\end{figure}

\textcolor{cyan}{静矩}(单位:$m^3$、$mm^3$)的定义式为:
\begin{equation}
	\begin{cases}
		S_x=\int_A ydA\\
		S_y=\int_A xdA
	\end{cases}
\label{jj}
\end{equation}

\textcolor{cyan}{形心坐标公式}为:
\begin{equation}
	\begin{cases}
		\bar{x}=\frac{\int_A ydA}{A}\\
		\bar{y}=\frac{\int_A xdA}{A}
	\end{cases}
	\label{xx}
\end{equation}

由公式\ref{jj}、\ref{xx},得
	\begin{gather}%多行公式用gather环境
	\bar{x}=\frac{S_y}{A} \Rightarrow S_y=A\bar{x}\\	
	\bar{y}=\frac{S_X}{A} \Rightarrow S_x=A\bar{y}
	\end{gather}

实际题目中出现的比较多的是规则图形组合截面的静矩和形心坐标,由定义式易得:
\begin{gather}
	S_x=\sum_{i=1}^n A_i\bar{y}_i \qquad S_y=\sum_{i=1}^n A_i\bar{x}_i\\
	\bar{x}=\frac{\sum\limits_{i=1}^n A_i\bar{y}_i}{\sum\limits_{i=1}^n A_i} \qquad \bar{y}=\frac{\sum\limits_{i=1}^n A_i\bar{x}_i}{\sum\limits_{i=1}^n A_i}
\end{gather}	
		
\subsection{极惯性矩、惯性矩和惯性积}
先上图:
\begin{figure}[htp]%设置figure环境
	\centering
	\includegraphics[width = 6cm ]{jgxj}
\end{figure}

\textcolor{cyan}{极惯性矩}(单位:$m^4$、$mm^4$,对点)的定义式为:
\begin{equation}
	I_p=\int_A\rho^2dA
	\label{jgxj}
\end{equation}

可由公式\ref{jgxj}推导出实心圆和空心圆对圆心O的极惯性矩为:
\begin{figure}[htp]%设置figure环境
	\centering
	\includegraphics[width = 12cm ]{2gy}
\end{figure}

\begin{equation}%\mbox用于在equation环境下加中文
\mbox{实心圆:} I_p=\frac{\pi D^4}{32} \qquad \mbox{空心圆:} I_p=\frac{\pi}{32}(D^4-d^4)
\end{equation}

\textcolor{cyan}{惯性矩}(单位:$m^4$、$mm^4$,对轴)的定义式为:
\begin{equation}
\begin{cases}
I_x=\int_A y^2dA\\
I_y=\int_A x^2dA
\end{cases}
\label{gxj}
\end{equation}

由公式\ref{jgxj}、\ref{gxj}知:
\begin{equation}
	I_p=\int_A(x^2+y^2)dA=I_x+I_y
\end{equation}

即截面对一点的极惯性矩,等于截面对以该点为原点的任意两正交坐标轴的惯性矩之和。

可由公式\ref{gxj}推导出矩形和平行四边形对其形心轴的惯性矩为:
\begin{figure}[htp]%设置figure环境
	\centering
	\includegraphics[width = 12cm ]{2g4}
\end{figure}
\begin{equation}
	I_x=\frac{bh^3}{12} \qquad I_y=\frac{hb^3}{12}
\end{equation}

圆对其形心轴的惯性矩为:
\begin{figure}[htp]%设置figure环境
	\centering
	\includegraphics[width = 6cm ]{y}
\end{figure}
\begin{equation}
	I_x=I_y=\frac{\pi D^4}{64}
\end{equation}

\textcolor{cyan}{惯性积}(单位:$m^4$、$mm^4$,对两轴)的定义式为:
\begin{equation}
I_{xy}=\int_A xydA
\end{equation}

\textcolor{cyan}{惯性半径}的定义式为:
\begin{equation}
i_x=\sqrt{\frac{I_x}{A}} \qquad i_y=\sqrt{\frac{I_y}{A}}
\end{equation}





























\newpage
%参考文献
\bibliographystyle{unsrt}%这个style根据引用顺序,plain根据作者姓名排序

\bibliography{document}



\end{document}