\documentclass[10pt,a4paper]{ctexart}
\usepackage{ctex}
\usepackage{titlesec}%调节标题格式
\titleformat*{\section}{\zihao{3}\bfseries\centering}
\titleformat*{\subsection}{\zihao{4}\bfseries}
\titleformat*{\subsubsection}{\zihao{4}\bfseries}
\usepackage[utf8]{inputenc}
\usepackage[T1]{fontenc}
\usepackage{amsmath}
\usepackage{pifont}
\renewcommand\thefootnote{\ding{\numexpr171+\value{footnote}}}
\newcommand\Emph{\textbf}
\usepackage{amssymb}
\usepackage{makeidx}
\usepackage{graphicx}
\graphicspath{{figures/}}%figures环境
\usepackage{float}
\usepackage{booktabs}
\usepackage{multirow}
\usepackage[section]{placeins}
\usepackage{geometry}
\geometry{hcentering}
\geometry{left=2cm,right=2cm,top=3cm,bottom=3cm}%设置左右页边距
\setcounter{secnumdepth}{4}%设置目录标题深度为4
\setcounter{tocdepth}{4}
\usepackage{color}
\usepackage{tabularray}
\usepackage{hyperref}%加高亮框框,但容易和其他包冲突,故放在后面
\begin{document}
	\begin{titlepage}
\centering
\zihao{1}鼠鼠的材力期末\protect\footnotemark%脚注符号与内容分开,这是脚注符号
\\
\vspace*{\baselineskip}
\zihao{-4}by:一只学土木的鼠鼠
	\footnotetext{edited by \LaTeX}%脚注符号与内容分开,这是脚注内容
	\newpage
	\end{titlepage}
 %\setlength{\parindent}{0pt}%取消所有段落首行缩进
 \noindent%取消该段首行缩进
 \zihao{1}前面的话

\vspace*{1ex}%设置段落间的垂直间距
\zihao{-4}这个材料力学的期末小册子是基于孙训方的《材料力学(\uppercase\expandafter{\romannumeral1}%这罗马数字也太tm难打了吧
 )》(第五版)\cite{孙训方2009}
 整理出来的,纯粹是鼠鼠为了期末复习和顺便重拾\LaTeX 而整理,如有真正想学好材料力学的朋友建议啃书本,鼠鼠只是想期末及格。
 
 而且这玩意儿我预感会通篇公式,几乎没有图片。本来鼠鼠是打算给必要的地方放上Axglyph、PowerPoint等软件画的矢量图,但实际操作还是不熟悉。等后面鼠鼠有时间了会捣鼓捣鼓\LaTeX 写论文的工作流,会给这篇加上矢量图的。
 
 鼠鼠来打脸啦,本来想着就整理点公式就结束,结果有些地方是不得不放上图片,不然很多点真的不好理解。大概有一半的图是鼠鼠自己使用Axglyph做的矢量图(鼠鼠尽量做得好看点),其他的图片都是老师们的PPT截取下来的图片,这里非常感谢哈斯亚提·哈里丁\cite{哈斯亚提·哈里丁2022}老师和赵文\cite{赵文}老师的教学和课件内容让这本小册子迸发出新的活力,还有b站up主@土木光头强\cite{b站up主@土木光头强}老师的精彩的教学视频让我受益匪浅。
 
 再说一点,这里面很多的内容其实并非出自课本,而是通过鼠鼠自己的理解加进去的俏皮话,便于理解。但这样很可能会出现用词不恰当,定义不严谨的情况,鼠鼠觉得严谨的学术用语和便于理解的大白话还是存在差异的。但是对于这篇用语并不严谨的期末小册子,鼠鼠尽量会在每个公式前加上该公式的适用条件,毕竟考试时公式都用错了鼠鼠这敲了这俩星期的代码算是白忙活了。
 
 遗憾的是,本书的第八章和第九章没有时间整理了,鼠鼠还有一周多的时间就考试了,现在刚整理完第七章还有几张CAD期末作业没画QAQ,不说了,鼠鼠画图去了。
\thispagestyle{empty}%不标出页码
 \newpage
\thispagestyle{empty}%放在插入目录前
\tableofcontents
\thispagestyle{empty}
\setcounter{page}{0}
\newpage
\setcounter{secnumdepth}{-1}%设置section章节不标号
\section{第一章:绪论及基本概念}
P.S.:经典大学课本第一章绪论。鼠鼠暂时没看到有什么重点,看到了鼠鼠会补充

P.P.S.:这册子里\textcolor{red}{出现在正文里的所有公式号}都能点击直达哦,打印出来的当我没说
\section{第二章:轴向拉伸和压缩}
这章可以说是材料力学中基础中的基础,第二节的截面法求轴力和画截面图更是后面所有内容的根基。
\subsection{轴力及轴力图}
轴力即作用线与杆轴线重合的力。在我们所学的范围内,杆件所受的拉力和压力绝大多数为轴力。\textcolor{red}{符号规定:}拉力为正,压力为负,一般计算先设为正,如果计算为负即为压力。

\textcolor{cyan}{截面法画轴力图:}三步走:截开、代替(用内力代替弃去部分)、平衡
可能会出大题,比较简单:先整体受力分析求支反力(求外力),再用截面法求内力(有时候右边比左边好算,具体情况具体分析),作图即可。
\subsection{应力}
应力分两种:正应力和切应力,正应力引发长度的改变,切应力引发角度的改变。本章主要是正应力的计算,其实可以完全把正应力当作压强看待,其计算公式为
\begin{equation}
	\sigma=\frac{F_N}{A}
	\label{zyl}
\end{equation}
单位为Pa,其中A为横截面面积。符号与轴力保持一致(拉正压负)。

切引力的\textcolor{red}{符号规定}要注意:对截面一点产生顺时针力矩的切应力为正,反之为负。

当发生\textcolor{red}{轴向拉压}时,斜截面上的正应力和切应力有公式解决,先如图示(不得不画图了)
\begin{figure}[htp]%设置figure环境
	\centering
	\includegraphics[width = 8cm ]{xjm}
\end{figure}

公式如下

\begin{equation}
	\begin{cases}
		\sigma_\alpha=p_\alpha cos\alpha =\sigma_0 cos^2\alpha\\
		\tau_\alpha=p_\alpha sin\alpha  = \frac{\sigma_0}{2}sin2\alpha
	\end{cases}
\end{equation}

$\sigma_0=\frac{F}{A}$是拉杆在横截面($\alpha=0$)上的正应力,由公式知切应力的最大值在与轴线成$45^{\circ}$角的斜截面上,大小为$\tau_{max}=\frac{\sigma}{2}$
\subsection{许用应力}
简单说就是杆件所受最大应力不应超过许用应力,公式为
\begin{equation}
	\sigma_{max}\leqslant\left[\sigma\right]%数学环境打出括号
\end{equation}
\par 若为轴向拉压,加上正应力公式,即公式\ref{zyl}%交叉引用的命令是ref,别忘了
\begin{equation}
	\sigma_{max}=\frac{F_{N,max}}{A}\leqslant\left[\sigma\right]
\end{equation}
\par
由它可以得到强度计算的三种类型:

\begin{table}[h]
	\centering
	\begin{tblr}{|c|c|}%好好看看tabularray是怎么用的
		\hline
		强度校核 & $\sigma_{max}=\frac{F_{N,max}}{A}\leqslant\left[\sigma\right]$ \\
		\hline
		截面选择 & $A\geqslant\frac{F_{N,max}}{\left[\sigma\right]}$\\
		\hline
		计算许可负载 & $F_{N,max}\leqslant A\left[\sigma\right]$\\
		\hline
	\end{tblr}
\end{table}


\subsection{横向线应变、纵向线应变、泊松比及胡克定律}
别管别的,把横纵向的区别搞清楚先:
\begin{figure}[htp]%设置figure环境
	\centering
	\includegraphics[width = 6cm ]{hzx}
\end{figure}

\textcolor{cyan}{线应变}为单位长度上的伸长和缩短(压缩和膨胀)。由定义得纵向线应变$\varepsilon$和横向线应变$\varepsilon'$为
\begin{equation}
\varepsilon=\frac{\Delta l}{l} \qquad%两公式间间隔
  \varepsilon'=\frac{\Delta l}{l}
  \label{dy}
\end{equation}

线应变的\textcolor{red}{符号规定}要注意:纵向伸长为正反之为负;横向压缩为正反之为负。

\textcolor{cyan}{泊松比}(横向变形因数)为\textcolor{red}{横向线应变$\varepsilon'$}与\textcolor{red}{纵向线应变之$\varepsilon$}之比(别比反了!),数值随材料而异。
\begin{equation}
 \nu=\frac{\varepsilon'}{\varepsilon} \Rightarrow \varepsilon'=-\nu \varepsilon
 \label{psb}
\end{equation}

\textcolor{cyan}{胡克定律}的原始表达式为:
\begin{equation}
\Delta l=\frac{F_N l}{EA}
\label{hkdl}
\end{equation}

其中E被称作材料的弹性模量,数值随材料而定。EA被称作杆件的拉伸(压缩)刚度,是衡量杆件抗拉(压)变形能力的,越大越强。

由公式\ref{dy}、\ref{hkdl},得胡克定律的应力-应变表达式:
\begin{equation}
\sigma=E\varepsilon
\end{equation}
\subsection{弹性应变能及弹性应变能密度}
\textcolor{cyan}{弹性应变能}为弹性体所受力而变形时所积蓄的能量。表达式为:
\begin{equation}
	V_{\varepsilon}=\frac{F^2_N l}{2EA}
\end{equation}

联立公式\ref{hkdl},得:
\begin{equation}
	V_{\varepsilon}=\frac{EA}{2l}\Delta l^2
\end{equation}

\textcolor{cyan}{弹性应变能密度}为单位体积的弹性应变能,即:
\begin{equation}
	v_{\varepsilon}=\frac{V_{\varepsilon}}{V}=\frac{\frac{1}{2}F\Delta l}{Al}=\frac{1}{2}\sigma\varepsilon
\end{equation}

联立公式\ref{hkdl},得:
\begin{equation}
	v_{\varepsilon}=\frac{E\varepsilon^2}{2}=\frac{\sigma^2}{2E}
\end{equation}
	 
\subsection{材料在拉伸和压缩时的力学性能}
这节就两理解性的公式,后面全是过程叙述,鼠鼠捡几个重点稍微说说。

\textcolor{cyan}{伸长率}$\delta$和\textcolor{cyan}{断面收缩率}$\psi$的公式为
\begin{equation}
	\delta=\frac{l_1-l}{l}\times 100\% \qquad \psi=\frac{A-A_1}{A}\times 100\%
\end{equation}

通常$\delta>5\%$的材料被称为\textcolor{cyan}{塑性材料},$\delta<5\%$的材料被称为\textcolor{cyan}{脆性材料}。

力学性能为材料受力时在强度和变形方面所表现出来的性能。

在\textcolor{cyan}{低碳钢的拉伸实验}中,由应力-伸长量图像可将拉伸过程分为四个阶段,如图所示:
\begin{figure}[htp]%设置figure环境
	\centering
	\includegraphics[width = 8cm ]{dtgls}
\end{figure}

\uppercase\expandafter{\romannumeral1}为\textcolor{red}{弹性阶段},满足胡克定律,比例极限$\sigma_p$对应点A(请自行放大,确实不太清楚),弹性极限$\sigma_e$对应点B。两者区别在于弹性极限是在弹性范围内的,即形变可以完全恢复;但比例极限是有可能超过了弹性形变的范围,仅仅是数值上符合胡克定律\cite{匿名用户2017}。

\uppercase\expandafter{\romannumeral2}为\textcolor{red}{屈服阶段},此阶段应变显著增加,但应力基本不变(屈服现象)。屈服极限$\sigma_s$对应D点。

\uppercase\expandafter{\romannumeral3}为\textcolor{red}{强化阶段},此阶段材料抵抗变形的能力有所增强。强度极限$\sigma_b$对应点G,为最大名义应力。

\uppercase\expandafter{\romannumeral4}为\textcolor{red}{局部变形阶段},试件上出现急剧的局部横截面收缩现象(颈缩现象),直至试件断裂。

\textcolor{cyan}{冷作硬化现象}指在强化阶段卸载后,材料的比例极限提高,塑性降低的现象。过程不再详细叙述,只需记住这样后比例极限$\sigma_p$提高,塑性变形降低,强度极限$\sigma_b$不变。

后面(课本P33及以后)就是一堆没什么规律的文字叙述了。自己看看书,鼠鼠在这里不做赘述。

\subsection{应力集中(概念)}
\textcolor{cyan}{应力集中}指杆件突然变化而引起的应力局部骤然增大的现象。用\textcolor{cyan}{理论应力集中系数}$K_{t\sigma}$描述应力集中程度:
\begin{equation}
	K_{t\sigma}=\frac{\sigma_{max}}{\sigma_{nom}}
\end{equation}

其中$\sigma_{max}$为杆件外形局部不规则处的最大局部应力,$\sigma_{nom}$为横截面的平均应力。

考虑应力集中的情况如下表所示:
\begin{table}[h]
	\centering
	\begin{tblr}{|c|c|}%好好看看tabularray是怎么用的
		\hline
		不考虑应力集中 & 塑性材料、静荷载;非均匀的脆性材料,如铸铁\\
		\hline
		考虑应力集中 & 均匀的脆性材料或塑性差的材料;动荷载\\
		\hline	
	\end{tblr}
\end{table}
\newpage

\section{附录\uppercase\expandafter{\romannumeral1}:截面的几何性质}
本章涉及大量的公式推导及数学计算,鼠鼠为追求短平快在此只列出必要的定义、公式及结论。对公式推导感兴趣的话可以翻翻书本。

\subsection{静矩和形心位置}
先上图:
\begin{figure}[htp]%设置figure环境
	\centering
	\includegraphics[width = 6cm ]{jjxx}
\end{figure}

\textcolor{cyan}{静矩}(单位:$m^3$、$mm^3$,对轴)的定义式为:
\begin{equation}
	\begin{cases}
		S_x=\int_A ydA\\
		S_y=\int_A xdA
	\end{cases}
\label{jj}
\end{equation}

\textcolor{cyan}{形心坐标公式}为:
\begin{equation}
	\begin{cases}
		\bar{x}=\frac{\int_A ydA}{A}\\
		\bar{y}=\frac{\int_A xdA}{A}
	\end{cases}
	\label{xx}
\end{equation}

由公式\ref{jj}、\ref{xx},得
	\begin{gather}%多行公式用gather环境
	\bar{x}=\frac{S_y}{A} \Rightarrow S_y=A\bar{x}\\	
	\bar{y}=\frac{S_X}{A} \Rightarrow S_x=A\bar{y}
	\end{gather}

实际题目中出现的比较多的是规则图形组合截面的静矩和形心坐标,由定义式易得:
\begin{gather}
	S_x=\sum_{i=1}^n A_i\bar{y}_i \qquad S_y=\sum_{i=1}^n A_i\bar{x}_i\\
	\bar{x}=\frac{\sum\limits_{i=1}^n A_i\bar{y}_i}{\sum\limits_{i=1}^n A_i} \qquad \bar{y}=\frac{\sum\limits_{i=1}^n A_i\bar{x}_i}{\sum\limits_{i=1}^n A_i}
\end{gather}	
		
\subsection{极惯性矩、惯性矩和惯性积}
先上图:
\begin{figure}[htp]%设置figure环境
	\centering
	\includegraphics[width = 6cm ]{jgxj}
\end{figure}

\textcolor{cyan}{极惯性矩}(单位:$m^4$、$mm^4$,对点)的定义式为:
\begin{equation}
	I_p=\int_A\rho^2dA
	\label{jgxj}
\end{equation}

可由公式\ref{jgxj}推导出实心圆和空心圆对圆心O的极惯性矩为:
\begin{figure}[htp]%设置figure环境
	\centering
	\includegraphics[width = 12cm ]{2gy}
\end{figure}

\begin{equation}%\mbox用于在equation环境下加中文
\mbox{实心圆:} I_p=\frac{\pi D^4}{32} \qquad \mbox{空心圆:} I_p=\frac{\pi}{32}(D^4-d^4)
\label{yjgxj}
\end{equation}

\textcolor{cyan}{惯性矩}(单位:$m^4$、$mm^4$,对轴)的定义式为:
\begin{equation}
\begin{cases}
I_x=\int_A y^2dA\\
I_y=\int_A x^2dA
\end{cases}
\label{gxj}
\end{equation}

由公式\ref{jgxj}、\ref{gxj}知:
\begin{equation}
	I_p=\int_A(x^2+y^2)dA=I_x+I_y
\end{equation}

即截面对一点的极惯性矩,等于截面对以该点为原点的任意两正交坐标轴的惯性矩之和。

可由公式\ref{gxj}推导出矩形和平行四边形对其形心轴的惯性矩为:
\begin{figure}[htp]%设置figure环境
	\centering
	\includegraphics[width = 12cm ]{2g4}
\end{figure}
\begin{equation}
	I_x=\frac{bh^3}{12} \qquad I_y=\frac{hb^3}{12}
	\label{jg}
\end{equation}

圆对其形心轴的惯性矩为:
\begin{figure}[htp]%设置figure环境
	\centering
	\includegraphics[width = 6cm ]{y}
\end{figure}
\begin{equation}
	I_x=I_y=\frac{\pi D^4}{64}
	\label{yg}
\end{equation}

\textcolor{cyan}{惯性积}(单位:$m^4$、$mm^4$,对两轴)的定义式为:
\begin{equation}
I_{xy}=\int_A xydA
\end{equation}

\textcolor{cyan}{惯性半径}的定义式为:
\begin{equation}
i_x=\sqrt{\frac{I_x}{A}} \qquad i_y=\sqrt{\frac{I_y}{A}}
\end{equation}

\textcolor{cyan}{惯性矩和惯性积的平行移轴公式}为:
\begin{figure}[htp]%设置figure环境
	\centering
	\includegraphics[width = 4.1cm ]{pxyz}
\end{figure}

\begin{equation}
	\begin{cases}
		I_x=I_{x_c}+a^2A\\
		I_y=I_{y_c}+b^2A\\
		I_{xy}=I_{x_cy_c}+abA
	\end{cases}
\end{equation}

\textcolor{cyan}{组合截面的惯性矩(或惯性积)}等于其各组成部分对同一坐标轴的惯性矩(或惯性积)之和。即:
\begin{equation}
	\begin{cases}
		I_x=\sum\limits_{i=1}^n I_{x_i}\\
		I_y=\sum\limits_{i=1}^n I_{y_i}\\
		I_xy=\sum\limits_{i=1}^n I_{x_iy_i}
	\end{cases}
\end{equation}
\newpage
\section{第三章:扭转}
本章开始就有点抽象了,鼠鼠们做好被薄纱的准备(bushi)。

\subsection{传动轴的外力偶矩}
传动轴的转速n(r/min)、功率P(kW)及在其上作用的\textcolor{cyan}{外力偶矩}$M_e$(N$\cdot$m)之间的关系为:

\begin{figure}[htp]%设置figure环境
	\centering
	\includegraphics[width = 8cm ]{dl}
\end{figure}

\begin{equation}
	M_e=9550\frac{P}{n}
\end{equation}

\subsection{扭矩及扭矩图}
\textcolor{cyan}{扭矩}T指圆轴受纽时其截面上的内力偶矩。

注意扭矩的\textcolor{red}{符号规定}:
\begin{figure}[htp]%设置figure环境
	\centering
	\includegraphics[width = 8cm ]{njfh}
\end{figure}

T的大小依然可以用截面法确定,三步走:截开、代替及平衡。

有了大小和正负,就可以按照轴力图的方法做出扭矩图,不再细锁。

\subsection{薄壁圆筒的扭转}
本节鼠鼠复习了半天也没完全整明白,课后老师也没有留作业,鼠鼠暂且认为本节内容不算重点,但却是后面等直圆杆的基础,而等直圆杆是重点中的重点。

\begin{figure}[htp]%设置figure环境
	\centering
	\includegraphics[width = 10cm ]{bbyt}
\end{figure}

薄壁圆筒受扭时,两端界面之间相对转过的圆心角被称为\textcolor{cyan}{相对扭转角}$\varphi$。表面上正方格子倾斜的角度,即直角的改变量被称为\textcolor{cyan}{切应变}$\gamma$,如上图所示:

由几何关系(鼠鼠没搞懂这几何关系怎么来的,书上也没仔细说QAQ)得:
\begin{equation}
	\gamma=\frac{\varphi r}{l}
\end{equation}

通过一系列数学推导,可得出\textcolor{cyan}{薄壁圆筒横截面上切应力计算公式}得:
\begin{figure}[htp]%设置figure环境
	\centering
	\includegraphics[width = 8cm ]{bbqyl}
\end{figure}
\begin{equation}
	\tau=\frac{T}{r_oA}=\frac{T}{2\pi r_o^2\delta}
\end{equation}

通过实验得到\textcolor{cyan}{剪切胡克定律}:当外力偶矩在某一范围内时,相对扭转角$\varphi$与外力偶矩$M_e$,(在数值上等于扭矩T)之间成正比,即:
\begin{equation}
	\tau=G\gamma
	\label{jq}
\end{equation}

G(单位:Pa)被称为材料的切变模量。

\subsection{等直圆杆的扭转}
先上图:
\begin{figure}[htp]%设置figure环境
	\centering
	\includegraphics[width = 8cm ]{dzyg}
\end{figure}

刚刚写了很多我觉得必要的公式推导,想想还是删掉了。因为这并不符合我写这玩意儿的初衷。

当等直圆杆受扭时,横截面周边上各点处切应力最大,即\textcolor{cyan}{等直圆杆最大切应力公式}为:
\begin{equation}
	\tau_{max}=\frac{Tr}{I_p}
\end{equation}

$r$代表横截面半径,$I_p$代表横截面的极惯性矩,$T$代表横截面上的扭矩。

若用$W_p$代表$I_p/r$,则有:
\begin{equation}
\tau_{max}=\frac{T}{W_p}
\end{equation}

$W_p$被称作\textcolor{cyan}{扭转截面系数},单位为$m^3$

鼠鼠觉得可以先背两个圆的极惯性矩结论,即公式\ref{yjgxj},$W_p$只需除以其半径即可。故不再赘述。

\textcolor{cyan}{相对扭转角}$\varphi$为:
\begin{equation}
	\varphi=\frac{Tl}{GI_p}
	\label{nzj}
\end{equation}

其中$l$为两横截面之间的距离,$G$为材料的切变模量。

和轴向拉压一样,扭转时杆件内也会积蓄\textcolor{cyan}{应变能},为:

\begin{equation}
	V_\varepsilon=\frac{T^2l}{2GI_p}
\end{equation}

联立公式\ref{nzj},得:
\begin{equation}
	V_\varepsilon=\frac{GI_p}{2l}\varphi^2
\end{equation}

\subsection{切应力互等定理}
直接给结论,这部分证明鼠鼠一头雾水QAQ。
\begin{figure}[htp]%设置figure环境
	\centering
	\includegraphics[width =6cm ]{hddl}
\end{figure}

\textcolor{cyan}{切应力互等定理}指两相互垂直平面上的切应力$\tau$和$\tau'$数值相等,且均指向(或背离)该两平面的交线。

\subsection{斜截面上应力公式}
经过一系列证明,得\textcolor{cyan}{斜截面上应力公式}QAQ:
\begin{figure}[htp]%设置figure环境
	\centering
	\includegraphics[width =10cm ]{xjmq}
\end{figure}

\begin{equation}
	\sigma_{\alpha}=-\tau sin2\alpha \qquad \tau_{\alpha}=\tau cos2\alpha
\end{equation}

由上述公式易知:

\begin{tblr}{l}
	1、& 单元体的四个侧面($\alpha=0^{\circ}$和$\alpha=90^{\circ}$)上切应力的绝对值最大\\
	2、&
	$\alpha=-45^{\circ}$和$\alpha=+45^{\circ}$截面上切应力为零,而正应力绝对值最大,其值为:\\
\end{tblr}
\begin{equation}
	\sigma_{-45^{\circ}}=\sigma_{max}=+\tau \qquad
	\sigma_{+45^{\circ}}=\sigma_{min}=-\tau
\end{equation}

\newpage

\section{第四章:弯曲应力}
本章可能看起来比扭转直观,但实际计算比扭转可要复杂多了,计算一直是鼠鼠的软肋$\cdots \cdots$扯远了哈哈,本章的重点在剪力图和弯矩图(后面简称“双图”)的画法及正切应力的计算。双图的两种画法里第二种利用规律及微分关系更重要,但第一种列剪力弯矩方程的方法也要知道。鼠鼠会在篇幅中加以体现。

\subsection{剪力和弯矩·剪力图和弯矩图}
如果说轴力是纵向内力,那么剪力就是横向内力(这个说法不甚严谨,但是对我们理解剪力挺有帮助的)。如下图所示(虚线部分为截面法截去部分):
\begin{figure}[htp]%设置figure环境
	\centering
	\includegraphics[width =10cm ]{jlwj}
\end{figure}

和轴力一样,\textcolor{cyan}{剪力}$F_S$和横向外力$F_A$相平衡。而\textcolor{cyan}{弯矩}$M$与$F_A$和$F_S$所形成的力矩相平衡。

剪力和弯矩的\textcolor{red}{符号规定}就尤为重要了:课本上的文字规定略显生涩,咱直接看图:

\begin{figure}[htp]%设置figure环境
	\centering
	\includegraphics[width =8cm ]{wjfh}
\end{figure}

对于剪力的符号:左上右下为正,左下右上为负(可用顺时针与逆时针简记,如红色旋转符号所示,注意这只是简记符号,不是力矩方向)。因为平时做题我们一般都会用到截面法,去掉截取部分后看剪力在所留部分的哪边,按照上面的符号规定定符号。

对于弯矩的符号就比较简单了:画完剪力方向,把剪力方向往反方向一拉就是弯矩方向。下侧受拉为正,上侧受拉为负。研究对象同样是所留部分。

对于它俩的计算,三个字:截面法,不做赘述。

双图的画法前面提到过:第一种就是就是借助截面法求剪力弯矩方程做函数图像,这个方法咱看看课本上例题理解下过程就行,遇到复杂的弯曲情况计算量就太大了。下面鼠鼠详细聊聊第二种:

首先,我们通过一系列的数理过程得到荷载$q$、剪力$F_S$和弯矩$M$之间的微分关系为:
\begin{equation}
	q=\frac{dF_S}{x} \qquad F_S=\frac{dM}{dx}
\end{equation}

由上述微分关系易得荷载的积分就是剪力,剪力的积分就是弯矩。那么我们的作图步骤就很明晰了,一般题目给的就是荷载,我们先做出剪力图,就可以通过剪力图得到弯矩图的变化规律。具体如图:
\begin{figure}[htp]%设置figure环境
	\centering
	\includegraphics[width =18cm ]{jwtgl}
\end{figure}

这里强调两点:

\begin{tblr}{l}
	1.&双图都是从左端0开始,右端0结束\\
	2.&弯矩图是正在下,负在上(土木是这样的,机械貌似是反过来的)
\end{tblr}

\subsection{弯曲正应力及强度条件}
首先先交代清楚。下面的公式从严格的数理证明来说仅适用于剪力为零,弯矩为常量的梁的情况(即纯弯曲情况)。但在工程实践中在特殊情况下(一般题目也会满足这个情况)其计算既有剪力又有弯矩的梁(即横力弯曲情况)误差是满足工程中的精度要求的。

\begin{figure}[htp]%设置figure环境
	\centering
	\includegraphics[width =10cm ]{zxz}
\end{figure}

弯曲变形时,由于变形的连续性,中间必有一层纵向线无长度改变,称为\textcolor{cyan}{中性层},中性层和横截面的交线称为\textcolor{cyan}{中性轴}。如上图所示:

经过一系列的数理证明,得\textcolor{cyan}{横截面上的任一点处的正应力}$\sigma$为:
\begin{equation}
	\sigma=\frac{My}{I_z}
\end{equation}

其中$M$是横截面上的弯矩;$I_z$是横截面对中性轴z的惯性矩;$y$为求应力点的纵坐标。

由上式知,当$y$达到最大时,$\sigma$最大,即:
\begin{equation}
	\sigma_{max}=\frac{My_{max}}{I_z}
\end{equation}

若令$W_z=I_z/y_{max}$,有
\begin{equation}
	\sigma_{max}=\frac{M}{W_z}
\label{wz}
\end{equation}

$W_z$被称为\textcolor{cyan}{弯曲截面系数},同扭转截面系数一样,鼠鼠还是觉得可以先背矩形和圆的惯性矩,即公式\ref{jg}、\ref{yg},分别除去$h/2$、$D/2$即可得到$W_z$。

弯曲正应力强度条件即为最大工作正应力$\sigma_{max}$不得超过材料的许用弯曲正应力$\left[\sigma\right]$,即$\sigma_{max}\leqslant\left[\sigma\right] $,由公式\ref{wz}得:
\begin{equation}
       \frac{M_{max}}{W_z}\leqslant\left[\sigma\right]
\end{equation}

\subsection{弯曲切应力及强度条件}
鼠鼠们请再次做好被薄纱的准备QAQ。如想了解弯曲切应力推导过程,鼠鼠推荐个b站up主@土木光头强的视频,这一节内容的bv号为BV18y4y127BP。

说回正题,本节只需要掌握矩形截面的弯曲切应力的公式及最大式的应用,薄壁圆筒和圆截面公式最好记住(万一考了呢)。如果真想考工字形$\cdots \cdots$就鼠鼠所在的学校来说应该是刻意不想让鼠鼠们及格的节奏。但因为每个学校的情况不同,鼠鼠会把这些公式都列出来。虽然应该不要求掌握但还是要注意下各个符号所代表的含义。

经过一系列的数理证明,得\textcolor{cyan}{矩形截面}等直梁在对称弯曲时横截面上任意一处切应力为:
\begin{equation}
	\tau=\frac{F_SS^{\ast}_z}{I_zb}
\end{equation}

\begin{figure}[htp]%设置figure环境
	\centering
	\includegraphics[width =3.5cm ]{jxjm}
\end{figure}

$F_S$为横截面上的剪力,$I_z$为整个横截面对其中性轴的惯性矩,$S^{\ast}_z$为横截面上距中性轴为$y$的直线以外部分的面积对中性轴的\textcolor{red}{静矩},$b$为矩形宽度,$\tau$的方向与剪力$F_S$的方向相同。如上图所示:

当弯曲切应力达到最大值时,有:
\begin{equation}
	\tau_{max}=\frac{3F_S}{2A}
\end{equation}

式中,$A=bh$,为矩形截面面积。

\textcolor{cyan}{薄壁环形截面}最大弯曲切应力为:
\begin{equation}
		\tau_{max}=2 \frac{F_S}{A}
\end{equation}

\begin{figure}[htp]%设置figure环境
	\centering
	\includegraphics[width =6cm ]{ytwqy}
\end{figure}

式中,$A=\frac{\pi}{4}\left[(2r_0+\delta)^2-(2r_0-\delta)^2\right]=2\pi r_0\delta$,代表环形截面面积。

\textcolor{cyan}{圆截面}最大弯曲切应力为:
\begin{equation}
	\tau_{max}=\frac{4F_S}{3A}
\end{equation}

\begin{figure}[htp]%设置figure环境
	\centering
	\includegraphics[width =4cm ]{cir}
\end{figure}

式中,$A=\pi d^2/4$,为圆截面面积。

同许用弯曲正应力一样,弯曲切应力的强度条件为最大弯曲切应力不得超过于许用弯曲切应力,即$\tau_{max}\leqslant\left[\tau\right]$。

\newpage

\section{第五章:梁弯曲时的位移}
这章$\cdots \cdots$鼠鼠尽自己最大所能整理点东西吧。

\subsection{挠度及转角}
先上图:
\begin{figure}[htp]%设置figure环境
	\centering
	\includegraphics[width =8cm ]{nd}
\end{figure}

横截面形心(即轴线上的点)在垂直于$x$轴方向的线位移$\omega$,称为该截面的\textcolor{cyan}{挠度};横截面对其原来位置的角位移$\theta$, 称为该截面的\textcolor{cyan}{转角}。梁变形后的轴线称为\textcolor{cyan}{挠曲线},建个系挠曲线便摇身一变变成
\textcolor{cyan}{挠曲线方程}。挠曲线方程对$x$的一阶导数即为
\textcolor{cyan}{转角方程}$\theta$。

\subsection{挠曲线的近似微分方程及其积分}
看到“近似”两字没有?和弯曲应力里的公式一样,下面提到的公式也是由特殊条件通过数理过程严格证明,但是在工程实践中误差很小,故可以用来计算一般情况。

\textcolor{cyan}{挠曲线的近似微分方程}为:
\begin{equation}
\omega''=-\frac{M(x)}{EI}
\end{equation}

由上式知,这是挠曲线方程的二阶导数。若想得到挠曲线方程需将上式积两次分。一般我们使用下面的变形式进行积分:
\begin{equation}
EI\omega''=-M(x)
\label{nbx}
\end{equation}

积一次得$\omega'$,又因为挠曲线方程对$x$的一阶导数即为转角方程$\theta$,可由此得转角方程。每次积分后都会出现常数,两次积分出现两次常数。但我们可以通过梁挠曲线的边界条件列方程得出常数大小,如固定端处挠度和转角均等于零等。

\textcolor{cyan}{叠加原理}梁在几项荷载(如集中力、集中力偶或分布力)同时作用下某一横截面的挠度或转角,就分别等于每项荷载单独作用下该截面的挠度或转角的叠加。就提一嘴,只能希望别考难题$\cdots\cdots$

梁的\textcolor{cyan}{刚度条件}可表达为:
\begin{equation}
\begin{cases}
	\frac{\omega_{max}}{l}\leqslant\left[\frac{\omega}{l}\right]\\
	\theta_{max}\leqslant\left[\theta\right]
\end{cases}
\end{equation}

\subsection{弯曲应变能}
直接给出全梁的弯曲应变能公式为:
\begin{equation}
V_{\varepsilon}=\int_l \frac{M^2(x)}{2EI}dx
\end{equation}

由公式\ref{nbx}知,上式可写作
\begin{equation}
	V_{\varepsilon}=\frac{EI}{2}\int_l (\omega'')^2dx
\end{equation}

\newpage

\section{第六章:简单的超静定问题}
超静定问题简单来说就是仅通过静力学方程解决不了的问题,怎么解?加方程!从期末考试的角度来看本章仅需掌握拉压超静定问题,而所附加的方程被称作变形几何相容方程。拉压超静定的变形几何相容方程就两个方面:胡克定律和几何关系。由于没有特别的公式,鼠鼠就不在细嗦,把课本上的两个例题弄懂即可。来小亮给他整个活,草!走!忽略!

\section{第七章:应力状态和强度理论}
这一章算是对前面所学的所有关于应力知识的归一,重点是平面应力状态的几个公式和广义胡克定律。空间应力状态和四大强度理论要有所了解。

\subsection{斜截面上的应力及应力圆}
若单元体有一对平面上的应力等于零,即不等于零的应力分量均处于同一坐标平面内,则称为\textcolor{cyan}{平面应力状态}。一般用如下图所示的小单元体来表示平面应力状态。
 \begin{figure}[htp]%设置figure环境
 	\centering
 	\includegraphics[width =9cm ]{pmyl}
 \end{figure}

接下来所有的公式都是基于平面应力状态的前提进行推导的。

首先再次明确各物理量的\textcolor{red}{符号规定}:
\begin{table}[h]
	\centering
	\begin{tblr}{|c|c|}%好好看看tabularray是怎么用的
		\hline
		 方位角$\alpha$& 以x轴正方向为基准,逆时针转向为正,顺时针为负\\
		\hline
		正应力$\sigma$ & 拉应力为正,压应力为负\\
		\hline
		切应力$\tau$	& 以与x轴垂直的切应力为基准,对单元体内点的矩为顺时针为正,反之为负。\\
		\hline
	\end{tblr}
\end{table}

由平衡公式推导得任意$\alpha$截面上的应力分量为;
\begin{gather}
	\sigma_{\alpha}=\frac{\sigma_x+\sigma_y}{2}+\frac{\sigma_x-\sigma_y}cos2\alpha-\tau_xsin2\alpha \label{1}\\
	\tau_{\alpha}=\frac{\sigma_x-\sigma_y}{2}sin2\alpha+\tau_xcos2\alpha \label{2}
\end{gather}

由公式\ref{1}、\ref{2}可得:
\begin{equation}
	\left(\sigma_{\alpha}-\frac{\sigma_x+\sigma_y}{2}\right)^2+\tau_{\alpha}^2=\left(\frac{\sigma_x-\sigma_y}{2}\right)^2+\tau_x^2
\end{equation}

这个方程在$\sigma-\tau$直角坐标系内的轨迹是一个圆,其圆心位于$\sigma$轴上,横坐标为$\frac{\sigma_x+\sigma_y}{2}$,半径为$\sqrt{\left(\frac{\sigma_x-\sigma_y}{2}\right)^2+\tau_x^2}$,该圆习惯上被称为\textcolor{cyan}{应力圆},如图所示:
 \begin{figure}[htp]%设置figure环境
	\centering
	\includegraphics[width =11cm ]{yly}
\end{figure}

怎么用好应力圆捏?先利用已知的应力$\sigma_x$$\tau_x$和$\sigma_y$$\tau_y(=-\tau_x)$标出两点,两点一条直线拉过去就是应力圆的直径。记住应力圆需要转动$2\alpha$才能得到$\alpha$斜截面的情况。

\subsection{主应力与主平面}
\textcolor{cyan}{主平面}为切应力等于零的截面,而\textcolor{cyan}{主应力}为主平面上的正应力。由定义易得主应力对应应力圆与$\sigma$轴相交处。 而单元体上从$x$平面转到$\sigma_1$主平面的转角$\alpha_0$为顺时针转向,按规定为负值。则由应力圆得:
\begin{equation}
	tan(-2\alpha_0)=\frac{\tau_x}{\frac{1}{2}(\sigma_x-\sigma_y)}
\end{equation}

则方位角为:
\begin{equation}
	2\alpha_0=arctan(\frac{-2\tau_x}{\sigma_x-\sigma_y})
\end{equation}

\subsection{广义胡克定律}
所谓\textcolor{cyan}{广义胡克定律}就是把横向线应变也纳入纵向线应变的考虑范围。在\textcolor{red}{平面应力状态}下的广义胡克定律为:
\begin{figure}[htp]%设置figure环境
	\centering
	\includegraphics[width =4cm ]{gyhk}
\end{figure}

\begin{equation}
\begin{aligned}
	\varepsilon_x=\frac{1}{E}(\sigma_x-\nu\sigma_y)\\
	\varepsilon_y=\frac{1}{E}(\sigma_y-\nu\sigma_x)\\
	\varepsilon_z=-\frac{\nu}{E}(\sigma_x+\sigma_y)\\
	\gamma_{xy}=\frac{1}{G}\tau_{xy}
\end{aligned}
\end{equation}

若已知\textcolor{red}{平面应力状态}下两个主应力$\sigma_1$、$\sigma_2$,得:

\begin{equation}
	\begin{aligned}
	\varepsilon_1=\frac{1}{E}(\sigma_1-\nu\sigma_2)\\
	\varepsilon_2=\frac{1}{E}(\sigma_2-\nu\sigma_1)\\
	\varepsilon_3=-\frac{\nu}{E}(\sigma_1+\sigma_2)
	\end{aligned}
\end{equation}

\subsection{四大强度理论}
如下表所示,了解即可:
\begin{table}[h]
	\centering
	\begin{tblr}{|c|c|c|}%好好看看tabularray是怎么用的
		\hline
		第一强度理论& 最大拉应力理论 & $\sigma_1\leqslant\left[\sigma\right]$\\
		\hline
		第二强度理论 & 最大伸长线应变理论 & $\sigma_1-\nu(\sigma_2+\sigma_3)\leqslant\left[\sigma\right]$\\
		\hline
		第三强度理论	& 最大切应力理论 & $\sigma_1-\sigma_3\leqslant\left[\sigma\right]$\\
		\hline
		第四强度理论 & 形状改变能密度理论 &$\sqrt{\frac{1}{2}\left[(\sigma_1-\sigma_2)^2+(\sigma_2-\sigma_3)^2+(\sigma_3-\sigma_1)^2\right]}\leqslant\left[\sigma\right]$\\
		\hline
	\end{tblr}
\end{table}

















\newpage
%参考文献
\bibliographystyle{unsrt}%这个style根据引用顺序,plain根据作者姓名排序

\bibliography{document}



\end{document}